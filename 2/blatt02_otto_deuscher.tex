\documentclass{article}

\usepackage[utf8]{inputenc}
\usepackage[ngerman]{babel}
\usepackage{enumitem}
\usepackage{listings}
\usepackage{lstautogobble}

\lstset{literate=
    {Ö}{{\"O}}1
    {Ä}{{\"A}}1
    {Ü}{{\"U}}1
    {ß}{{\ss}}1
    {ü}{{\"u}}1
    {ä}{{\"a}}1
    {ö}{{\"o}}1
    {~}{{\textasciitilde}}1
}
\newcommand{\java}{\lstinline[language=Java]}

%Blatt Nr
\newcommand{\blattNr}{2}
\setcounter{section}{\blattNr}

\begin{document}
    \title{EidI Übungsblatt \blattNr}
    \author{Jonas Otto, Marco Deuscher}
    \date{5. November 2017}
    \maketitle

    \section*{}
        \subsection{}
            \begin{lstlisting}[autogobble]
                Setze zahl=random(1,20)
                Setze tipp=0;
                Setze i=1;
                Solange (i<=3) Wiederhole:
                    Ausgabe: "Geben sie ihren Tipp ein:"
                    Lese die Eingabe in tipp ein.
                    Falls (tipp>zahl)
                        Ausgabe:
                            "Ihr Tipp ist größer als die gesuchte Zahl."
                        i=i+1;
                    Falls(tipp<zahl)
                        Ausgabe:
                            "Ihr Tipp ist kleiner als die gesuchte Zahl."
                        i=i+1;
                    Falls(tipp==zahl)
                        Ausgabe:
                            "Sie haben die Zahl erraten." +
                            "Die gesuchte Zahl war " + zahl;
                        Beende Programm;
                Ausgabe: "Die gesuchte Zahl war " + zahl +
                    ". Sie waren nicht in der Lage, diese zu erraten."
                Beende Programm;
            \end{lstlisting}

        \subsection{}
            \begin{enumerate}[label=(\roman*)]
                \item double: Es werden Nachkommastellen für cent benötigt
                \item char: Eignet sich um ein einzelnes Symbol zu speichern
                \item String: Geeignet um sowohl Zahlen als auch Buchstaben zu
                    speichern
                \item String: Anders als bei Zahlen gibt es hier keine
                    Größenbeschränkung
                \item boolean: Wahrheitswert mit zwei Möglichkeiten (Tür auf/Tür
                    zu)
                \item double: Fließkommazahl mit höherer Genauigkeit
            \end{enumerate}

        \subsection{}
            \begin{lstlisting}
                (A&&!A) ||  !(7!=9 && 21>42)
            \end{lstlisting}
            \java{(A&&!A)} ist immer wahr. Kann hier
            ignoriert werden (Da oder Verknüpfung).\\
            \java{(7!=9)} ist immer wahr. Kann hier
            ignoriert werden (Da und Verknüpfung).\\
            \java{(21>42)} ist immer falsch.\\
            \java{!(21>42)} ist immer wahr.\\
            Die gesamte Zeile kann durch \java{true}
            Ersetzt werden.
\end{document}
