\documentclass{article}

\newcommand{\blattNr}{3}
\usepackage[utf8]{inputenc}
\usepackage[ngerman]{babel}
\usepackage{enumitem}
\usepackage{listings}
\usepackage{lstautogobble}

\lstset{literate=
    {Ö}{{\"O}}1
    {Ä}{{\"A}}1
    {Ü}{{\"U}}1
    {ß}{{\ss}}1
    {ü}{{\"u}}1
    {ä}{{\"a}}1
    {ö}{{\"o}}1
    {~}{{\textasciitilde}}1,
    basicstyle=\small\ttfamily
}
\newcommand{\java}{\lstinline[language=Java]}

\setcounter{section}{\blattNr}

\newcommand{\uebungsblattTitel}{
    \title{EidI Übungsblatt \blattNr}
    \author{Jonas Otto, Marco Deuscher}
    \maketitle
}


\begin{document}
    \uebungsblattTitel

    \section*{}
        \subsection{}
            \begin{lstlisting}[autogobble]
                Setze z = 1
                Während z <= 100
                    Setze t = (int) (z/2)
                    Während t > 1
                        Wenn z % t == 0, dann:
                            Abbruch der Schleife
                        t = t - 1
                    Wenn t == 1, dann:
                        Gib aus: z
                    z = z + 1
            \end{lstlisting}
        \subsection{}
            \begin{align}
                \begin{split}
                    3/5*-2&==3+5-2\\
                    3/5*(-2)&==(3+5)-2\\
                    (3/5)*(-2)&==(3+5)-2\\
                    ((3/5)*(-2))&==((3+5)-2)
                \end{split}
            \end{align}

            \begin{align}
                \begin{split}
                    \text{false}\ ||\ 8\ \%\ -3 &== 7 * 2\ \&\&\ \text{true}\\
                    \text{false}\ ||\ 8\ \%\ (-3) &== (7 * 2)\ \&\&\ \text{true}\\
                    \text{false}\ ||\ (8\ \%\ (-3)) &== (7 * 2)\ \&\&\ \text{true}\\
                    (\text{false}\ ||\ (8\ \%\ (-3))) &== ((7 * 2)\ \&\&\ \text{true})
                \end{split}
            \end{align}

        \subsection{}
            % TODO 3.3
            \begin{lstlisting}[language=Java, autogobble]
            System.out.println(0.3+0.3+0.3);
            \end{lstlisting}
                Ausgabe: 0.8999999999999999\\
                Begründung: 0.3 wird jeweils als double interpretiert, da Java aber Dezimalzahlen
                    nicht exakt darstellen kann, wird nicht genau 0.9 sondern eine Annäherung ausgegeben.
            \begin{lstlisting}[language=Java, autogobble]
                System.out.println('a'+'b'+'c'+"!");
            \end{lstlisting}
                Ausgabe: 294!\\
                Begründung: 'a','b','c' sind char und werden aufgrund des + Operators als Integer behandelt
                            entsprechend dem ASCII-Code ist a=97, b=98, c=99\\
                            "!"\ ist ein String und wird auch als solcher ausgegeben
            \begin{lstlisting}[language=Java, autogobble]
                System.out.println(9/2);
            \end{lstlisting}
                Ausgabe: 4\\
                Begründung: Es werden zwei Ganzzahlen durcheinander geteilt, folglich wird auch eine
                            Ganzzahl als Ergebnis ausgegeben. Die Nachkommastellen werden abgeschnitten.
            \begin{lstlisting}[language=Java, autogobble]
                System.out.println("Rechnung: " + 3 + -1 + 5);
            \end{lstlisting}
                Ausgabe: Rechnung: 3-15\\
                Begründung: Da zuerst der String "Rechnung: "\ ausgegeben wird, wird alles folgende ebenfalls
                            als String behandelt. Wollte man mit den Zahlen noch eine Rechenoperation ausführen
                            müsste man entsprechend Klammern setzen.

        \subsection{}
            \lstinputlisting[language=Java]{src/GrumpyCat/GrumpyCat.java}

\end{document}
